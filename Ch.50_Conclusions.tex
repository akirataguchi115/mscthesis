\chapter{Conclusions\label{conclusions}}
% primary objective of this study
The primary objective of this study was to conduct an extensive review of public licenses in software engineering and offer support to both practitioners and reserachers. The study addressed the following research questions:
\textbf{RQ1: How many public software licenses are there in the top five software license listing cites?} The results of this study

\textbf{RQ2: How much is there disagreement in the shortcode names between different public software licenses listing sites?}

\textbf{RQ3: How many public licenses in software engineering does there exist?}


% conclusions from each rq

\section{Future research}
% adopting a clear baseline
Future research and industry efforts should start at adopting a clear baseline. 

% why agplv3 is the best license

% dockerc cla, sspl, post-open

% make cla easier with gpg / joplin easy cla sign

% LICENSE highlighting.js

% what kind of efforts and why

% what this thesis has provided (REWRITE THIS TEMPLATE)
In conclusion, this thesis has provided a systematic review of the current state of research on public licenses in software engineering. Through a systematic literature review, we have identified the main approaches employed, evaluated the quality and evidence level of the research, tried to analyze the common factors of successful approaches and identified the categories and limitations of the approaches. The results of this study indicate that the field of green in software engineering is still immature, with a lack of common standards, definitions, and tooling. However, this also provides an opportunity for further research and development in this area. It is our hope that the findings of this study will serve as a starting point for future research and industry efforts to build the standards and tooling required for more thorough analysis and information sharing, leading to more sustainable software engineering practices. 