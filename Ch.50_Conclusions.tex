\chapter{Conclusions\label{conclusions}}
% primary objective of this study
The primary objective of this study was to conduct an extensive review of public licenses in software engineering and offer support to both practitioners and reserachers. The study addressed the following research questions:

% conclusions from rq1
\textbf{RQ1: How many licenses are there in the top five software license listing cites?} The results of this study reveal 607 public licenses listed from the SPDX, 38 from the DFSG, 200 from the FSF, 96 from the OSI and 116 from the GNU project with the four latter sites providing most difficulities in fetching and parsing the shortcodes and full license texts with additional problems in overlapping and contradicting information. This resulted in 1057 total licenses.

% conclusions from rq2
\textbf{RQ2: How much is there disagreement in the shortcode names between different public software licenses listing sites?} From the 1057 original licenses, some 339 licenses were removed as duplicates using their shortcodes. Reasons for this initial amount of duplicates were discussed and lay background and reasoning for the amount of disagreement in the shortcode names between our license listing sites.

% conclusions from rq3
\textbf{RQ3: How many public licenses in software engineering does there exist?} After a manual review of the literature gathered and processed, the study found that there are some 594 unique public software licensed listed that can be obtained systematically. 


\section{Future research}
% adopting a clear baseline
Future research and industry efforts should start at adopting a clear baseline. The definition of free software, open source and everything in between and ultimately the insignificance of the categorization regarding these public software licenses. It is important for the future research to distinguish the terms used yet it is also as important to know that the for example not all free software licenses offer the same freedoms.

% next research project
In terms of research efforts, a research project that would concentrate on trade-offs and especially the harmful impacts of software projects that do not use a strong copyleft public software license could have significant effect. This project could benefit from the metrics that contain the most used public software licenses and the most significant public software licensed software, and the effects of using that license with the consideration of the impact of using a different license. 

\section{Future industry efforts}
% dockerc cla, sspl, post-open

% why agplv3 is the best license

% what this thesis has provided
In conclusion, this thesis has provided a systematic review of the current state of research on public licenses in software engineering.Through a systematic literature review, we have identified the amount of unique public software licenses and the methods of obtaining these licenses in a systematic way. The results of this study indicate that the field of public licenses in software engineering is still immature, with a lack of common standards, definitions, and tooling. However, this also provides an opportunity for further research and development in this area. It is our hope that the findings of this study will serve as a starting point for future research and industry efforts to build the standards and tooling required for more thorough analysis and information sharing, leading to more sustainable software engineering practices.