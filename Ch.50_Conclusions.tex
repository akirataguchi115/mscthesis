\chapter{Conclusions\label{conclusions}}
% conclusions from rq1
\textbf{RQ1: How many public licenses in software engineering does there exist?} The results of this study reveal 607 public licenses listed from the SPDX, 38 from the DFSG, 200 from the FSF, 96 from the OSI and 116 from the GNU project with the four latter sites providing most difficulities in fetching and parsing the shortcodes and full license texts with additional problems in overlapping and contradicting information. This resulted in 1057 total licenses. After a manual review of the literature gathered and processed, the study found that there are some 594 unique public software licensed listed that can be obtained systematically.

% conclusions from rq2
\textbf{RQ2: How consistent are the naming conventions for public licenses in software engineering?} From the 1057 original licenses, some 339 licenses were removed as duplicates using their shortcodes. Reasons for this initial amount of duplicates were discussed and lay background and reasoning for the amount of disagreement in the shortcode names between our license listing sites. 

\section{Future research}
% adopting a clear baseline
Future research efforts should start at adopting a clear baseline. The definition of free software, open source and everything in between and ultimately the insignificance of the categorization regarding these public software licenses. It is important for the future research to distinguish the terms used yet it is also as important to know that the for example not all free software licenses offer the same freedoms. While fixing all the problems in public software licenses singlehandedly is not feasible, establishing a solid baseline for terminology is a start.

% next research project
In terms of research efforts, a research project that would concentrate on trade-offs and especially the harmful impacts of software projects that do not use a strong copyleft public software license could have significant effect. This project could benefit from the metrics that contain the most used public software licenses and the most significant public software licensed software, and the effects of using that license with the consideration of the impact of using a different license. 

\section{Future industry efforts}
% openwashing and post open
The author believes that future industry efforts should focus on trying to learn and understand the practical effects of each public software license used in a software project as a dependency or as an individiual software project licensed under a public software license. Due to the loose definition of open source companies like OpenAI have nothing to do with open source and large language models classes like Meta's Llama 4 are not open source although it is marketed as ''Open-Source''. Because open source is often seen as altruistic and generally speaking good for the image of a company, the industry continues to see ''openwashing'' just like greenwashing is used to market something as ''green'' when it in reality is not. As we have seen in this thesis' results, the definition of open source truly is too loose as it covers even licenses like \texttt{Beerware} and \texttt{JSON} where payments are in optional alcoholic beverages and restrictions are good and evil. This is why another one of the creators of the original Open Source Definition, Bruce Perens left the initiative \citep{register:perens-left-osi}. Perens has then proposed a successor to Open Source called Post Open Source. The paradigm combines new public software licenses, existing public software licenses and a practice where one singular organization would gather the money from larger corporations using post open software and distribute it to the contributors of the post open licensed software. It remains to be seen whether or not this will revolutionize the future industry of software engineering. 

% why agplv3 is the best license, watcom
When it comes to licensing new code from the ground up the near future industry should pay close attention to choosing a sustainable license. As mentioned earlier some of the licenses are not even meant to be court-proof. Licenses like the \texttt{GPL} are constantly trialed by fire in the court and are succesfully defending software freedom even today \citep{gplv2-court}. Although even more copyleft-oriented public software licenses like the \texttt{Watcom} exist, the author recommends licensing software source code under \texttt{AGPL-3.0-or-later} where possible due to its strong copyleft attributes and freedoms but mostly the court-proof nature of it as well.

% what this thesis has provided
In conclusion, this thesis has provided a systematic review of the current state of research on public licenses in software engineering.Through a systematic literature review, we have identified the amount of unique public software licenses and the methods of obtaining these licenses in a systematic way. The results of this study indicate that the field of public licenses in software engineering is still immature, with a lack of common standards, definitions, and tooling. However, this also provides an opportunity for further research and development in this area. It is our hope that the findings of this study will serve as a starting point for future research and industry efforts to build the standards and tooling required for more thorough analysis and information sharing, leading to more sustainable software engineering practices.