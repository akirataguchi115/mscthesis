\chapter{Introduction\label{intro}}

Public copyright licenses play a major part in software engineering. For example in open source there must be an appropriate public copyright license attached to the source code in order for open-source software to be freely available for possible modification and redistribution. Because open source is central to software engineering the licenses enabling open source must also be considered important in the same context.

Public copyright license is defined by Wikipedia with the following words \citep{wiki:publiclicenses}:
\begin{quote}
	''A public copyright license is a copyright license where the licensees are not limited. Examples include free content, open content, Creative Commons, free software and open source licences.''
\end{quote}

Understanding public copyright licenses can be difficult. This could stem from the legal nature of the license texts and the large number of already-existing public copyright licenses. The license texts usually favors correctedness over the readability for the developer. This is because the license text has to act as a valid legal instrument otherwise it cannot be endorsed \citep{ferguson2006gpl}. The lack of understanding of public copyright licenses leaves too much room for interpretation. In June 21, 2023 IBM's Red Hat seemingly violated a public copyright license, GNU General Public License version 2 or GPLv2 for short \citep{sfc:rhel} \citep{ibm:rhel}. This was an unpleasant surprise to the public since the project behind GPL, GNU initially attempted to ensure the users via the GPL have to the following three freedoms \citep{gnu:free}:
\begin{itemize}
	\item Freedom 1:	The freedom to study how the program works, and change it so it does your computing as you wish. Access to the source code is a precondition for this.
	\item Freedom2: The freedom to redistribute copies so you can help others
	\item Freedom 3:	The freedom to distribute copies of your modified versions to others. By doing this you can give the whole community a chance to benefit from your changes. Access to the source code is a precondition for this.
\end{itemize}

Regardless, IBM's Red Hat essentially rendered the previously public Red Hat Enterprise Linux, or RHEL for short, proprietary software. If the licenses would be more easily understood the proprietarization of RHEL would have been less of a surprise to the users.

On top of public copyright license details, software engineers in general have a tough time understanding the basic goals of public copyright licenses used in software engineering. In the instance of the RHEL incident it would not have been a big surprise to software engineers if they would have known about other licenses and what they try to achieve or how old is GPLv2 and why it has been succeeded by GPLv3.

This thesis' goal is to contribute into the solving these problems in a structured manner. First we state definitions and terminology used in the scope of this thesis. We go over the reasons why there does not exist consistent terminology in this area and why the conversely the definitions are the most stabile ones in this area. Second we take a deep dive into the multivocal literature through a systematic literature review. To make more information available, a mapping study connected to the terminology scope defined in the first step is needed. Third includes our own suggestions and basic knowledge for professionals and academics in the industry to enhance the understanding of public copyright licenses in software engineering. This step also includes discussion of the future research and contributes to stablizing the terminology and reinforcing the already-existing definitions in the academic field.

\section{Research goal, questions and contributions}
The secondary goal of this research is to conduct a systematic multivocal literature review of the current state of the public copyright licenses in software engineering, the evaluation of the them and the evidence level of the research. The research aims to provide a novel perspective on relevant licenses and to extract key findings through a rigorous literature review process. The research questions of the review are:
\begin{itemize}
	\item RQ1: How often do public copyright licenses in software engineering change?
	\item RQ2: How have the public copyright licenses in software engineering changed?
	\item RQ3: How long is the average public copyright license in software engineering?
	\item RQ4: What are the common reasons for version changes to public copyright licenses in software engineering?
\end{itemize}

Terms such as open source, source code, software freedom and other vocabulary must be defined in the scope of this thesis. Section \ref{sec:bg} will examine this plethora of of terminology and definitions and will be used to establish a sound basis for discussing this broad subject.

This study has two goals: to provide rigorous research on public copyright licenses to the academic field and to provide insights to the professional field of software engineering on public copyright licenses. The grand goal of this thesis is to raise awareness of the importance of public copyright licenses so that more licensers would make the correct choices based on their situations and needs in a mindful way.

\section{Thesis structure}
This thesis follows the IMRaD structure. Chapter 1 introduces the problem, this thesis' possible contributions and some further background. Chapter 2 goes over the process and the methods of the systematic multivocal literature review. Biggest part of the research happens here. Chapter 3 presents results to the research questions. Chapter 4 discusses implications for research. The chapter also discusses software engineering professionals in the thesis' context and the validity of the thesis' research. Chapter 5 concludes this thesis with the help of the research questions and the future of the research.

\section{Background and terminology of public copyright licenses}
literature review here
\label{sec:bg}