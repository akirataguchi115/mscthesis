\chapter{Introduction\label{intro}}

Public copyright licenses play a major part in software engineering. In the case of open source there must be an appropriate open-source license attached to the source code in order for open-source software to be freely available for possible modification and redistribution. Because open source is central to software engineering the licenses enabling open source must also be considered important in the same context.

Public copyright license is defined by Wikipedia:
\begin{quote}
	''A public copyright license is a copyright license where the licensees are not limited. Examples include free content, open content, Creative Commons, free software and open source licences.''
\end{quote} \citep{wiki:publiclicenses}

Using public copyright licenses can be difficult. This could stem from the legal nature of the license texts and the large number of already-existing public copyright licenses. The license texts favor correctedness over the understaning of the developer because the license text has to act as a valid legal instrument.

\section{Research goal, questions and contributions}

\begin{itemize}
	\item RQ1: How have public software licenses changed throughout the years?
	\item RQ2: Why have public software licenses changed throughout the years?
	\item RQ3: What is the speed of change in software licenses throughout the years?
	\item RQ4: What are the limitations for a rapidly changing software license?
	\item RQ5: How can we make it easier for developers to understand various public software licenses?
\end{itemize}

\section{Background and terminology of public copyright licenses}
literature review here

\section{Thesis structure}

Write how this thesis is going to contribute to the aforementioned problems.

Introduction 3 pages

Methods 10 pages

Results 10 pages

Discussion 6 pages

Conclusions 1 page
