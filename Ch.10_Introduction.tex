\chapter{Introduction\label{intro}}

Public copyright licenses play a major part in software engineering. For example in open source there must be an appropriate public copyright license attached to the source code in order for open-source software to be freely available for possible modification and redistribution. Because open source is central to software engineering the licenses enabling open source must also be considered important in the same context.

Public copyright license is defined by Wikipedia with the following words \citep{wiki:publiclicenses}:
\begin{quote}
	''A public copyright license is a copyright license where the licensees are not limited. Examples include free content, open content, Creative Commons, free software and open source licences.''
\end{quote}

Understanding public copyright licenses can be difficult. This could stem from the legal nature of the license texts and the large number of already-existing public copyright licenses. The license texts favor correctedness over the understaning of the developer because the license text has to act as a valid legal instrument otherwise it cannot be endorsed. The lack of understanding of public copyright licenses leaves too much room for interpretation. In June 21, 2023 IBM's Red Hat seemingly violated a public copyright license, GNU General Public License version 2 or GPLv2 for short \citep{sfc:rhel} \citep{ibm:rhel}. This was an unpleasant surprise to the public since the project behind GPL, GNU initially attempted to ensure the users via the GPL have to the following three freedoms \citep{gnu:free}:
\begin{itemize}
	\item Freedom 1:	The freedom to study how the program works, and change it so it does your computing as you wish. Access to the source code is a precondition for this.
	\item Freedom2: The freedom to redistribute copies so you can help others
	\item Freedom 3:	The freedom to distribute copies of your modified versions to others. By doing this you can give the whole community a chance to benefit from your changes. Access to the source code is a precondition for this.
\end{itemize}

Regardless, IBM's Red Hat essentially rendered the previously public Red Hat Enterprise Linux, or RHEL for short, proprietary software. If the licenses would be more easily understood the proprietarization of RHEL would not have been a surprise to anyone.

On top of public copyright license details, software engineers in general have a tough time understanding the basic goals of public copyright licenses used in software engineering. In the instance of the RHEL incident it would not have been a big surprise to software engineers if they would have known about other licenses and what they try to achieve or how old is GPLv2 and why it has been succeeded by GPLv3.

thesis' contribution to the solution:

\section{Research goal, questions and contributions}

\begin{itemize}
	\item RQ1: How have public software licenses changed throughout the years?
	\item RQ2: Why have public software licenses changed throughout the years?
	\item RQ3: What is the speed of change in software licenses throughout the years?
	\item RQ4: What are the limitations for a rapidly changing software license?
	\item RQ5: How can we make it easier for developers to understand various public software licenses?
\end{itemize}

\section{Background and terminology of public copyright licenses}
literature review here

\section{Thesis structure}

Write how this thesis is going to contribute to the aforementioned problems.

Introduction 3 pages

Methods 10 pages

Results 10 pages

Discussion 6 pages

Conclusions 1 page
