\chapter{Introduction\label{intro}}

\textcolor{olive}{2 other theses' 1.0 followed the following two forms setting, definition, problem, easier sub-problem, thesis' contribution to solution and setting, definition, observation, transferability, thesis' contribution to solution, thesis structure. i'll try the first one and let the supervisor let me know if i should try another approach for 1.0. 2/3 slrs that i read have some citation in the 1.0. Since proper and rigorous research on open source doesn't exist I won't include one at least now unless supervisor tells otherwise.
}

Licenses are in the core of open-source software. In order for open-source software to be freely available for possible modification and redistribution there must be an appropriate open-source license attached to the source code. Therefore the software license plays a central role on how the open-source software is developed and used now and in the future.

Open-source is defined by OSI in the following way: blah blah

\textcolor{olive}{start using another term for public software which includes everything else but trade secrets.}

Closed-source software slows down the advancement of the society and allows companies to ask for unsustainable amounts of money for software. Free software attempts to eliminate closed-source software. Free software leans heavily on the software licenses that juridically forbid closed-source derivatives of the software licensed with a free software license. Unlike modern software these licenses do not patch their loopholes frequently. One example of such a license failure is the RHEL incident where an operating system distribution licensed with free software was partially converted into closed source \citep{ferguson2006gpl}.

Write something about the state of the public software licenses here.


\section{Research goal, questions and contributions}

\begin{itemize}
	\item RQ1: How have public software licenses changed throughout the years?
	\item RQ2: Why have public software licenses changed throughout the years?
	\item RQ3: What is the speed of change in software licenses throughout the years?
	\item RQ4: What are the limitations for a rapidly changing software license?
	\item RQ5: How can we make it easier for developers to understand various public software licenses?
\end{itemize}

\section{Background and terminology of public software licenses}
literature review here

\section{Thesis structure}

Write how this thesis is going to contribute to the aforementioned problems.

Introduction 3 pages

Methods 10 pages

Results 10 pages

Discussion 6 pages

Conclusions 1 page
