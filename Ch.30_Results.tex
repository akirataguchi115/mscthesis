\chapter{Results\label{results}}
% information about chapter
This chapter employes the data extracted from the set of primary literature, available in this thesis' repository \citep{mscthesis} under the name of \texttt{stage1-licenses.md}, utilizing the methods outlined in \hyperref[methods]{Chapter 2} to address the research questions. Firstly, a summary of the general statistics collected and aggregated from the studies is presented. Following that, an analysis of the data is performed to provide answers to each of the research questions.

% note about publication year and literature amount
To begin with, the publication year was not limited and could not have been limited in a rigorous way. Almost all of the public software licenses came from different sources although they were listed in the five license listing sites. To give a rough estimate, one of the earliest public software license aiming for legal compliance was the original GPL from 1989 \citep{license-history}. The search was carried out by web scraping all of the licenses from the five license listing sites without any filters to the attributes of the licenses. The initial search results included 1057 public licenses, but after the exclusion and quality criteria of software-only license scope, the final resulting dataset was reached.

Given the large starting dataset, a simple statistical overview of the literature was generated and is presented in Figure citehere with the full list of papers available in this thesis' repository \citep{mscthesis} under the name of \texttt{stage1-licenses.md}.

\begin{figure}

\end{figure}

\section{Placeholder question (RQ1)}
\textcolor{red}{figures and literature identifier tables}
\section{Placeholder question (RQ2)}
\textcolor{red}{figures and literature identifier tables}
\section{Placeholder question (RQ3)}
\textcolor{red}{figures and literature identifier tables}
\section{Placeholder question (RQ4)}
\textcolor{red}{figures and literature identifier tables}

\textcolor{red}{essential statistics (figure)}
PUT THESE SOMEWHERE HERE SINCE THESE ARE MORE GOOD RESULTS THAN VALIDITY ISSUES 

\textcolor{red}{it's good to note that when bumping into the missing license of attpubliclicense which was from gnu, it turned out that the license listing site doesnt state the license content whatsoever. i had to just put the comment of the license into manual licenses.}

\textcolor{red}{mention how many missing/manual licenses there were. and from which site respectively}

\textcolor{red}{it could be a good idea to mention how many missing licenses came out of which sites. or it could be out of scope. i can just make a validity threat and say with face value that most of them were from FSF, GNU in that order. Python license seems just straight up an accident on FSF's side. scope is not to fix the documentation problems of the 5 organizations though so I'll leave it just like that and mention the possibility of it being just an accident.}

\textcolor{orange}{the eye-pass of 87 excluded licenses included cal1.0 and cal-combined which seem to need to be included. made inclusions.txt to manually include them although containing the words creative commons, since those license texts were licensed themselves with creative commons. will eyeball the stage-2 shortcodes as well just to see anything i know by heart is not public software license. cc-by-sa-japanese was included so i had to make a manual exclusions.txt to manually mark licenses that are not public software licenses.}

\textcolor{orange}{656 licenses after stage 2}


\textcolor{orange}{we relaized that we were trying to fix the problem we were trying to give awareness to in the thesis by trying to distinguish automatically what is a public software license in the sea of 600+ licenses.}

\textcolor{red}{do i need to discuss the results with alves et al 2010???}