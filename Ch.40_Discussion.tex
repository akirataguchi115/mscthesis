\chapter{Discussion\label{discussion}}
% indications
This research indicates that the field of public licenses in software engineering, meaning public software licenses, is at an early stage of development. The review of existing literature reveals a notable absence of common set of definitions and terminology in this field. Consequently, this often leads to work covering the same ground. Furthermore, the variability of terminology across different literature makes it challenging to compare and synthesize results effectively. Addressing this challenge will require the development of common measurement tools and frameworks for evaluating and comparing public licenses in software engineering. Such efforts could lead to the establishment of widely adopted standards for measuring and improving public software licenses.

% follow-up observation
That said, there is clear interest in public licenses in software engineering. Although the amount of purely academic literature on public software licenses is limited, the amount of grey literature on public software licenses is ever increasing. This is most likely due to the recent developments in the field regarding proprietarization noted in the \hyperref[intro]{Chapter 1}. 

% notable observation
A notable observation of our reserach is that the recent efforts in the industry have led to the development of Post Open Source not yet noted in the research literature reviewed \citep{register:poss}. Whether or not this paradigm will change the industry like open source and free software did will only be evident over time. The paradigm is explained shortly in \hyperref[conclusions]{Chapter 5}.

\section{Implications for research}
% how to generally improve scientific scene 1
To improve the maturity of research methods employed in the field of public licenses in software engineering, researchers should aim to use more rigorous and comprehensive research methods. This may involve using larger and more diverse data sets, developing more sophisticated measurement tools, and conducting experiments that are representative of real-world scenarios.

% how to generally improve scientific scene 2
Furthermore, researchers should strive to increase the transparency and reproducibility of their research by making their data and code openly available. This would enable other researchers to replicate and build upon their work, as well as facilitate the establishment of common standards and best practices.

% improve scientific scene multivocal vs academic
Finally, it is important for researchers to publish more articles regardless of the grey literature included in the papers. Because there is largely only grey literature published in the twenty-first century in the field, the next academic articles will be multivocal by default. The non-multivocal, academic articles will follow but only after there are systematic, academic and multivocal articles published for the former to build on. The results presented here are modest but by working together, researchers and industry professionals could produce more useful research regarding public licenses in software engineering.

\section{Implications for software engineering professionals}
% how to improve professional scene 1
Software engineering professionals should start by educating themselves of the basics of public licenses in software engineering and incorporating it into their design and development processes. They should be mindful or strive to mindful about the public licenses their third-party softwares are using and how it impacts their craft. Making a map of the public software licenses and their corresponding usecases might help plotting the larger picture.

% how to improve professional scene 2
However, it is important to acknowledge that the institutions should hold the greater responsibility of teaching the basics of public software licenses without getting too tangled up in history, politics or simply waving the field of as a form of human rights activism. The key focus points here being vocational schools, software engineering courses, college and university since these are the timestamps where most software engineers start to produce code that need to be licensed or require the use of a licensed piece of software.

% overall implications
Overall, the lack of public software license knowledge regarding software engineering professionals points to the need of more education regarding public software licenses and the practical effects stemming from the application of these licenses.

\section{Limitations and threats to validity}
The major limitation of this study is that the subjective results could not be validated by multiple researchers. In a systematic review, it is standard practice and highly recommended to have at least two, if not more, individuals independently conduct the review processes and then cross validating the findings. This would result in the possibility of comparing individual exclusion decisions and other decicions, thereby increasing the credibility of the study. However, in this study, the methodology was thoroughly documented, which allows us to assert with confidence that the study has an appropriate level of of validity.

As a work of single researcher, there is also a chance of inaccuracy and bias in the literature selection and filtering process. As much of the literature had to be reviewed manually and then included/excluded on a qualitative basis, this is a known limitation and a threat to validity. Multiple rounds of documented filtering and a clear paper trail of all decisions made keeps this threat in the acceptable levels.

\subsection{Limitations of literature selection for review}
Efforts were made to ensure the inclusion of comprehensive set of literature in the search process. This was achieved by setting the starting point of license listing sites to the Wikipedia article of the MIT license.

The first phase of filtering has some notable limitations starting with the two license listing websites: SPDX and DFSG. Since the material was gathered to a spreadsheet program the duplicates were removed using the short identifier the listing page was using. Let's look at this validity threat using an example. Suppose our spreadsheet program has acquired the public license with an identifier ''MIT''. The results of phase one will not include any other PCL marked with the identifier ''MIT''. In the worst case the identifier ''MIT'' could have actually been ''MIT-DFSG-edition'' but with the identifier of ''MIT''. Since there were so many public software licenses in phase one it would not have been possible to check the uniqueness of all removed duplicates. One of the reasons why this would not have been feasible is that the listing sites would fetch the public license contents from another webpage or at the second worst case, from another website. The worst case is that the URl is dead and we get HTTP 404. The amount of public software licenses, duplicates and the lack of already existing tools makes this problem multilayered. However this is the level of integrity we decided to finish our study with.

FSF's license listing introduced us to pick another limitation for the scope of this thesis. The license shortcoded as ''other'' was not a PCL but instead a hyperlink to another listing webpage that listed programs that the FSF has no yet managed to document the license which the program uses. Although the one of the programs called ''babl'' was licensed as with ''gplv3'' the amount of undocumented programs was over 5200 at the time of observation. For this reason we are excluding the public software licenses found indirectly from the category ''other''.

GNU project's listing site allowed us to use a shortcut of sorts which we will document here for the purposes of acknowleding the limitations of it. The table of contents at the listing site marked certain consequtive public software licenses as software public software licenses. On top of this the public software licenses were not organized into easily processable tables but rather in stacked on one another in rich text format. Although we decided to use regex on the HTMl file the included public software licenses were only the ones that were simply under the header ''Software licenses''. In the worst case scenario GNU project could have misinterpreted some public software licenses as non-software licenses thus making this thesis exclude them with a wrong reason. While from a quick glance and the existence of the other four license listing sites, we think it is still worth documenting when it comes to validity and the integrity of this thesis.

On top of too heavy filters we would also like to document the too light filters in the literature selection for review. We can see from \hyperref[appendix:a]{Appendix A} that for example public software licenses with the literature identifiers L777 and L780 are almost the same regarding the shortcoded identifiers: ''ZPL - 2.1'' and ''ZPL-2.1''. The duplicate removal would have been seemingly simple to execute on phase 1. However with the presence of over 700 pieces of literature we decided not to give special treatment to any potential set of duplicates. While it is most possible that OSI's ''ZPL - 2.1'' is equivalent exactly to SPDX's ''ZPL-2.1'' we could not be sure without looking at their contents. This could have resulted duplicate public software licenses in the literature selection for review but these type of duplicates are removed in phases 2 and 3 due to the public software licenses being read in full.

% Miscellaneous validity issues on literature selection 
To finish this subsection we will discuss some more minor validity issues that did not fit into \hyperref[results]{Chapter 3} but are regardless important to note for the integrity of the thesis. Stage three of the search process included a validity threat regarding the removal of duplicates. If two full license texts would seem duplicates we would check the two license listing sites' license pages for further investigation without using an internet archiver. This is a common validity threat on this thesis, that is not relying on an internet archiver on every source possible. Still, archiving more than a thousand license pages and accessing them would have been very slow process in terms of both archiving and accessing.

% why exclusion over inclusion
As can be seen in  \hyperref[methods]{Chapter 2} the regular expression string was only an exclusion filter. Using an inclusion and exclusion resulted in difficulities to match all of the public software licenses. In other words it eventually turned out to be faster to match the excludable licenses than the includable licenses. The validity threat lying here is that only using an exclusion filter implicates a majority of the public licenses in our dataset to be public software licenses. An example of difficult to include public software license is the \texttt{wtfpl} which includes no evidence of it being a public software license but rather a general public license. However because \texttt{wtfpl} is a largely used in software source code as can be seen in \hyperref[results]{Chapter 3}. Another examples to back up this choice in exclusion-only are the font licenses that are considered public software licenses. With the exceptions inflating the inclusion regular expression string we eventually decided to only use the aforementioned exclusion filter. Before the decision our inclusion string looked like this:
\begin{verbatim}
  (.*\b(source|software|program|code|module|public(s+)license|ware|
  (w+)ware)\b).*
\end{verbatim}

As mentioned earlier in the thesis the Wikipedia infobox order of license listing sites plays a heavy role in the literature selection. This manifests as a validity threat for example in removal of duplicates where the duplicates are removed from the lattermost listing site, giving a false sense of the majority of the public licenses coming from the formermost license listing sites like the SPDX. While this might be true due to the high volume of literature from the formermost license listing sites in order of the Wikipedia infobox it is still a threat to validity. Because of this choice in our scope the accuracy of the origins of the licenses in the search stages is not as high as it could be.

\textcolor{red}{. dejavu and dbg-3.0 were also two other licenses that contained a space. this might indicate that the space is an accident that its simply just not found from a license listing site x. its also good to note that the python script was decided to be an valid approach since many of the licenses were actually found with the shortcode from the licensedb scancode. fetching 700 licenses by hand would have had time and validity issues. wayback machine could have been used to do the actual searching as well. this is unfortunately a validity issue but at least the source is available in wayback machine.}

\textcolor{red}{For example licenses like CorkForkPL from FSF are just empty licenses. CorkForkPL is used however in MighTyD project but the license would have to be seen from a downloaded project or something like that. A new scope: only licenses that are one (1) click away from the initial license landing page can be copypasted to the manual licenses. just like the JahiaCSL has a URL on FSF for the license new location although the FSF page is empty.}

\textcolor{red}{licenses like MPL exist on FSF and GNU. it was not easily found from FSF (empty with links to programs using this) so it was gotten from GNU which was labeled as MPL1.1, which the FSF DOES have so i just boldly went with that. threat to validity.}

\textcolor{orange}{its good to note that systematic != automatic. our approach especially uses automation (python) to help the author use their human eye sight otherwise it would be more prone to error due to the large amount of licenses}

\textcolor{orange}{python script does not work on windows machine due to some os dependent path problems - validity threat}

\textcolor{red}{remember to document the validity L of human eye sight used majorly on third stage of search process. duplicate removal in tabs was done so that: i check the text if the n and n+1 and if they look pretty much the same i act and if the shortcodes look the sam i act.}

As mentioned in the beginning of this section, efforts were made to ensure the inclusion of comprehensive set of literature in the search process. However, as with all systematic literature reviews, a comprehensive manual review of all literature would have been a formidable task. Therefore, additional filtering was conducted. This filtering was carried out in two phases, starting with the application of inclusion/exclusion criteria, followed by a second phase focused on evaluating the nature of the public software licenses and conducting a manual review. As a result of this second phase, a set of literature were excluded following a critical appraisal, with documentation and reasoning provided for each section.

As such we can note that the literature selection was done in a sufficient manner.

\subsection{Limitations in data extraction}
% importance of data extraction
The process of data extraction holds great significance in a systematic literature review, as it has a direct impact on the transparency and rationale of the paper. The data extraction approach was shallow due to the data extraction form being relatively small. As mentioned above, not much data could be easily nor verifyiably extracted from our main grey literature, the five license listing sites. Despite the dilligent efforts to eliminate researcher bias, which is a common concern in interpretive methods, it was not feasible to replicate this work by another individual for cross-referencing purposes. However, the study's validity can still be considered appropriate, due to the transparent steps taken and the use of a short, but well-defined data extraction format.

% lack of measurements and tooling
We still note that because of the lack of common standardized measurements and tooling for them, a considerable amount of personal consideration had to be done to bring the research results of the primary literature into a comparative state.