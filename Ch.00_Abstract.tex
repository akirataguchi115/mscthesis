% \begin{abstract}{finnish}

% Tämä dokumentti on tarkoitettu Helsingin yliopiston tietojenkäsittelytieteen osaston opin\-näyt\-teiden ja harjoitustöiden ulkoasun ohjeeksi ja mallipohjaksi. Ohje soveltuu kanditutkielmiin, ohjelmistotuotantoprojekteihin, seminaareihin ja maisterintutkielmiin. Tämän ohjeen lisäksi on seurattava niitä ohjeita, jotka opastavat valitsemaan kuhunkin osioon tieteellisesti kiinnostavaa, syvällisesti pohdittua sisältöä.


% Työn aihe luokitellaan  
% ACM Computing Classification System (CCS) mukaisesti, 
% ks.\ \url{https://dl.acm.org/ccs}. 
% Käytä muutamaa termipolkua (1--3), jotka alkavat juuritermistä ja joissa polun tarkentuvat luokat erotetaan toisistaan oikealle osoittavalla nuolella.

% \end{abstract}

\begin{otherlanguage}{english}
\begin{abstract}
  \textbf{Context:} Public software licenses are central to the distribution of works in software engineering. For example in open source there must be an appropriate public software license attached to the source code in order for open-source software to be freely available for possible modification and redistribution. The large number of these public software licenses is one contributing factor in the difficulity of understanding how these licenses work.

  \textbf{Methods:} As the goal is to explore and evaluate different public licenses in software engineering, whilst also taking a look into the different sites listing these public software licenses, this study adopts a systematic literature review approach. The search strings, search process and other relevant information are meticulously documented and explored in each step of the research process.

  \textbf{Results:} 594 unique public software licenses were found from five different license listing sites. The found amount hints at the problem of too many public software licenses which might convert to the difficulity of understanding public software licenses, which again might lead to unexpected and unwanted legal agreements. Both research and industry have room for improvements such as new research using grey literature, use of \texttt{AGPL-3.0-or-later} or some other court-proven public software license and possibly supporting the new Post Open Source movement, that could revolutionaize the software industry.

  \textbf{Conclusions}: Future research efforts should start at adoping a clear baseline including the terminology. Future industry efforts should focus on trying to learn and understand the practical effects of the public software licenses being used.
\end{abstract}
\end{otherlanguage}

\section*{Acknowledgements}
ScanCode LicenseDB data is licensed under the Creative Commons Attribution License 4.0 (CC-BY-4.0) by nexB Inc. and others.

Thanks to Sami Nurmivaara for the thesis about green in software engineering. Mimicking the structure made it easy for me to write about my own subject within the area of software engineering. 

Thanks to professor Tomi Männistö and professor Mika Mäntylä.

Thanks to Dr. Richard Matthew Stallman and Bruce Perens for helping with the thesis and the smaller projects related to it.

Thanks to Iikka Hauhio and Joonas Jakobson for giving a hint for using a license database. Thanks to Rashid Harvey and Barunes Padhy for sending me software licensing related videos and news that kept me challenged. Thanks to def for borrowing GPT-4 back when it was not a commodity. 


Thanks to Suvi for supporting my work with the thesis from the beginning and enduring my mental health issues and being a good, caring mother.

Thanks to all my friends and family who were and are there for me.

Dedicated to Artemis.