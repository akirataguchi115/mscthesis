% \begin{abstract}{finnish}

% Tämä dokumentti on tarkoitettu Helsingin yliopiston tietojenkäsittelytieteen osaston opin\-näyt\-teiden ja harjoitustöiden ulkoasun ohjeeksi ja mallipohjaksi. Ohje soveltuu kanditutkielmiin, ohjelmistotuotantoprojekteihin, seminaareihin ja maisterintutkielmiin. Tämän ohjeen lisäksi on seurattava niitä ohjeita, jotka opastavat valitsemaan kuhunkin osioon tieteellisesti kiinnostavaa, syvällisesti pohdittua sisältöä.


% Työn aihe luokitellaan  
% ACM Computing Classification System (CCS) mukaisesti, 
% ks.\ \url{https://dl.acm.org/ccs}. 
% Käytä muutamaa termipolkua (1--3), jotka alkavat juuritermistä ja joissa polun tarkentuvat luokat erotetaan toisistaan oikealle osoittavalla nuolella.

% \end{abstract}

\begin{otherlanguage}{english}
\begin{abstract}
	Closed-source software slows down the advancement of the society and allows companies to ask for unsustainable amounts of money for software. Free software attempts to tackle this issue. Free software leans heavily on the software licenses that juridically forbid closed-source derivatives of the software licensed with a free software license.

	Unlike agile software projects of today these licenses do not patch their loopholes frequently. These loopholes cause exploitation of the licenses. One example of such a license failure is the RHEL incident where an operating system distribution licensed with a free license was partially was essentially rendered into proprietary software.

\end{abstract}
\end{otherlanguage}

\section*{Acknowledgements}
