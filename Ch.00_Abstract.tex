% \begin{abstract}{finnish}

% Tämä dokumentti on tarkoitettu Helsingin yliopiston tietojenkäsittelytieteen osaston opin\-näyt\-teiden ja harjoitustöiden ulkoasun ohjeeksi ja mallipohjaksi. Ohje soveltuu kanditutkielmiin, ohjelmistotuotantoprojekteihin, seminaareihin ja maisterintutkielmiin. Tämän ohjeen lisäksi on seurattava niitä ohjeita, jotka opastavat valitsemaan kuhunkin osioon tieteellisesti kiinnostavaa, syvällisesti pohdittua sisältöä.


% Työn aihe luokitellaan  
% ACM Computing Classification System (CCS) mukaisesti, 
% ks.\ \url{https://dl.acm.org/ccs}. 
% Käytä muutamaa termipolkua (1--3), jotka alkavat juuritermistä ja joissa polun tarkentuvat luokat erotetaan toisistaan oikealle osoittavalla nuolella.

% \end{abstract}

\begin{otherlanguage}{english}
\begin{abstract}
	Public copyright licenses (PCL) play a major part in software engineering. For example in open source there must be an appropriate PCL attached to the source code in order for open-source software to be freely available for possible modification and redistribution. Understanding PCLs can be difficult. This could stem from the legal nature of the license texts and the large number of already-existing PCLs. sub-problem here. thesis' contribution to solution here.
	
	Tell about the research method here.
	
	Tell about the results here.

	Tell about the discussion here.
\end{abstract}
\end{otherlanguage}

\section*{Acknowledgements}
