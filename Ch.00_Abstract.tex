% \begin{abstract}{finnish}

% Tämä dokumentti on tarkoitettu Helsingin yliopiston tietojenkäsittelytieteen osaston opin\-näyt\-teiden ja harjoitustöiden ulkoasun ohjeeksi ja mallipohjaksi. Ohje soveltuu kanditutkielmiin, ohjelmistotuotantoprojekteihin, seminaareihin ja maisterintutkielmiin. Tämän ohjeen lisäksi on seurattava niitä ohjeita, jotka opastavat valitsemaan kuhunkin osioon tieteellisesti kiinnostavaa, syvällisesti pohdittua sisältöä.


% Työn aihe luokitellaan  
% ACM Computing Classification System (CCS) mukaisesti, 
% ks.\ \url{https://dl.acm.org/ccs}. 
% Käytä muutamaa termipolkua (1--3), jotka alkavat juuritermistä ja joissa polun tarkentuvat luokat erotetaan toisistaan oikealle osoittavalla nuolella.

% \end{abstract}

\begin{otherlanguage}{english}
\begin{abstract}
  % change to Introduction, Methods and Results as per nurmivaara 2023
	\textbf{Context:} Public licenses are central to the distribution of works in software engineering. For example in open source there must be an appropriate PCL attached to the source code in order for open-source software to be freely available for possible modification and redistribution. Understanding public licenses can be difficult. This could stem from the legal nature of the license texts and the large number of already-existing public licenses. As a result some actions made within the boundaries of the public licenses may come as a surprise to the public.
	
	\textbf{Objective:} The primary goal of this research is to conduct a multivocal literature review of the current state of public licenses in software engineering, the evaluation of the them and the evidence level of the research. The research aims to provide a novel perspective on relevant licenses and to extract key findings through a rigorous literature review process. This study has two main viewpoints: to provide rigorous research on public licenses to the academic field and to provide insights to the professional field of software engineering on public licenses. The grand goal of this thesis is to raise awareness of the importance of public licenses so that more licensers would make the correct choices based on their situations and needs in a mindful way.
	
	\textbf{Method:} The search strategy examined 656 sources, found through websites that list public licenses and ad-hoc searches. Applying inclusion and exclusion criteria resulted in the selection of 656 sources, which made relevant contributions related to public licenses in software engineering.
	
	\textbf{Results:} 

	\textbf{Conclusions:} 
\end{abstract}
\end{otherlanguage}

\section*{Acknowledgements}
ScanCode LicenseDB data is licensed under the Creative Commons Attribution License 4.0 (CC-BY-4.0) by nexB Inc. and others.

much love to artemis, sami nurmivaara, prof männistö and prof mäntylä

thanks to def for borrowing gpt4. thanks to rashid and barunes for sending me software licensing related videos and news. thanks to iikka and joonas for giving a hint for using a license database. thanks to suvi for supporting my work with the thesis from the beginning and enduring my mental health issues.

thanks to stallman and perens for helping with the thesis and the sub projects related to that.